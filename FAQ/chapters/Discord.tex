Wenn ihr mit Discord arbeiten wollt, benötigt ihr eine Discord Client App mit der erstellt ihr einen Discord Account. Dabei wird wird Nickname, Email und Passwort benötigt. Ihr könnt dann sofort loslegen und den Link zu unserem Discod Server klicken und eich mit unserem Server verbinden. URL: \url{https://dc.wak-lab.org} Alternativ: \url{https://discord.gg/FUJuq4h}\\
\ \\
Nach der Registrierung kommt eine Bestätigungs-E-Mail in euer Postfach, das könnt ihr dann ja noch abschließen. Evtl. landet die Mail auch in eurem Spam Ordner. \\
\ \\
\begin{raggedright}
\begin{tabular}{|p{7,3cm}|p{7,3cm}|}
\hline
\textbf{Channel} & \textbf{Beschreibung}\\
\hline
WAK-LAB & Wak-Lab Themen\\
\cline{1-1}
\#wak-lab\_talk & \\
\cline{1-1}
\#events & \\
\cline{1-1}
\#offtopic & \\
\cline{1-1}
\#develop & \\
\hline
LAB TOPIC & Allgemeine Themen\\
\cline{1-1}
\#windows & \\
\cline{1-1}
\#linux & \\
\cline{1-1}
\#develop & \\
\cline{1-1}
\#funk\_radio\_sdr & \\
\cline{1-1}
\#3d-druck & \\
\cline{1-1}
\#smart-home & \\
\cline{1-1}
\#werkstatt & \\
\cline{1-1}
\#awareness & \\
\hline
ELEKTRONIK & Elektronik Themen\\
\cline{1-1}
\#electronik & \\
\cline{1-1}
\#arduino & \\
\cline{1-1}
\#esp8266\_und\_co & \\
\cline{1-1}
\#raspberry-pi & \\
\hline
WAK-LAB(INTERN, SPÄTER E.V.) & Interne Themen\\
\cline{1-1}
\#wak-lab\_verein & Geschlossene Gruppe\\
\cline{1-1}
\#wak-lab\_vorstand & Geschlossene Gruppe\\
\cline{1-1}
\#ankündigungen & Nächste Schritte\\
\cline{1-1}
\#abstimmungen & Nächste Schritte\\
\hline
\end{tabular}
\captionof{table}{Liste der Channels}
\label{tab:Channels}
\end{raggedright}


