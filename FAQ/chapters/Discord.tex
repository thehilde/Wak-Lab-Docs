Wenn ihr mit Discord arbeiten wollt, benötigt ihr eine Discord Client App mit der erstellt ihr einen Discord Account. Dabei wird wird Nickname, Email und Passwort benötigt. Ihr könnt dann sofort Loslegen und den Link zu unserem Discod Server Klicken und eich mit underem Server verbinden. URL: \url{https://dc.wak-lab.org} Alternativ: \url{https://discord.gg/FUJuq4h}\\
\ \\
Nach der Registrierung kommt eine Bestätigungs E-Mail in euer Postfach, das könnt ihr dann ja noch abschließen. Evtl. landet die Mail im Spam Ordner. \\

\ \\
\begin{raggedright}
\begin{tabular}{|p{7,3cm}|p{7,3cm}|}
\hline
\textbf{Channel} & \textbf{Beschreibung}\\
\hline
\#wak-lab\_talk & Gruppenchat\\
\hline
\#wak-lab\_verein & Geschlossene Gruppe\\
\hline
\#wak-lab\_vorstand & Geschlossene Gruppe\\
\hline
\#events & Gruppenchat\\
\hline
LAB TOPIC & Allgemeine Themen\\
\cline{1-1}
\#windows & \\
\cline{1-1}
\#linux & \\
\cline{1-1}
\#develop & \\
\cline{1-1}
\#funk\_radio\_sdr & \\
\cline{1-1}
\#3d-druck & \\
\hline
ELEKTRONIK & Elektronik Themen\\
\cline{1-1}
\#electronik & \\
\cline{1-1}
\#arduino & \\
\cline{1-1}
\#esp8266\_und\_co & \\
\cline{1-1}
\#raspberry-pi & \\
\hline
\end{tabular}
\captionof{table}{Liste der Channels}
\label{tab:Channels}
\end{raggedright}


